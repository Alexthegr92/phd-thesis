\chapter{Problem Definition and Proposed Solution}\label{ProbDef}
%	\item Naive approach. How would a naive approach behave and what problems occur.
%\item Considered Solutions. We consider a number of solutions to scalable real-time physics.
%\item Proposed Solution. We look at in more detail how our solution works and justification for its choice.
%\item Aura Calculation. A detailed look at the important aspect of AP, the aura calculation.
%\item Sub-optimal object hosting. We look at one the predicted challenge with AP, sub-optimal object hosting and discuss a solution which can be addressed in future work.
%\item Thrashing. A further predicted challenge with AP, this consists of two types of thrashing, two and three object thrashing, three object thrashing has been solved in this implementation of AP and we discuss solutions to two object thrashing.
%\item Islands. A solved challenge with AP, here we discuss our solution.
%\item Corner Case. AP has been discussed in terms of a single boundary between two servers up to this point, but in order for a general solution to be achieved, AP must be able to support boundaries between three or more servers, such as at corners.

In this chapter the requirements of a solution to scalable real-time physics and the challenges of distributing a real-time physics engine will be discussed, using a naive approach as an example. Some potential solutions to distributing real-time physics and their trade-offs will be considered. Finally, our proposed solution (AP) will be described, along with high-level overviews of the challenges with this approach and our implemented and proposed solutions.

\section{Solution Requirements}

In order for any solution to scalable real-time physics to maintain scalability, the extra processing overhead must be less than that gained by distributing the simulation workload. In addition, when scaling, a solution must maintain the correctness and fidelity of the simulation, i.e. errors should not be introduced.

%TODO: Discuss scalability
In general terms, distributing work between servers have the following overheads: the logic of determining messages to be sent; the processing overheard of sending the messages themselves; and the logic of handling received messages from other servers.
%For example, in a 
To achieve the greatest scalability, a solution must minimise overheard while maximising useful work done by each server.

The types of errors that can occur in distributed real-time physics simulations will now be discussed. 
Types of errors include:
\begin{itemize}
	\item Time-space inconsistencies, in which two different objects occupy the same space at the same time, but on different servers.
	\item Two objects occupying the same space at the same time, which can also lead to late collisions and missed collisions. 
	\item Diverging results and object state conflict, in which the same object has different states on different servers, which can be caused by collision response results differing on different servers.
	\item The same object having different states on different servers can lead to objects `jittering', as the object state is conflicting between the two servers. 
	\item Objects not being simulated (missed updates), which could occur when objects are migrating between servers. 
	\item Objects being duplicated and objects being lost. 
\end{itemize}

Examples and consequences of the above errors will now be described.

\subsection{Time-space inconsistencies}
Objects being simulated on different servers may end up occupying the same space and the same time, which is a physical inconsistency. This error could occur due to latency issues between servers or if a naive approach is taken to handling objects near server-region boundaries. For example, in a naive approach, object ownership may be determined by the centre of the object and no special consideration is given to objects near server-region boundaries. If two objects on different servers are overlapping the boundary at the same place, but their centres remain in their owning server's region, the two objects will have no interaction, despite occupying the same space at the same time. In a centralised version of the simulation, the two objects would collide and interact as soon as the two began to overlap with each other.
%If a solution is used in which 
%Objects may be unaware of objects on a different server

\subsection{Diverging results}\label{divergingResults}
PhysX is non-deterministic, as a result of this, repeating the same scenario with the exact same starting parameters does not necessarily lead to the exact same end state. The order in which collisions are resolved are randomised to achieve greater stability of the simulation.
If collision response is performed on different servers with the same group of objects, they will not lead to the same results. PhysX has an option for increased determinism, but as servers will have different states, as they will be hosting objects that the other server is not, this will not guarantee the exact same results. The results of a single differing collision response can create a significantly different state of objects, for example if a stack of cubes is dropped onto a flat, level plane, the cubes land is such a way where the stack is on the verge of collapsing. In one iteration, the stack of cubes may remain standing, in a second iteration, the stack of cubes may collapse.

As a consequence of non-determinism and diverging results, it is not possible to simply simulate the same objects on two different servers without some means of the two servers reaching consensus on the object state. In addition, it is not possible to measure correctness of a distributed simulation by comparing it with a single centralised simulation, i.e. run both solutions with the exact same starting parameters and compare the final state of the distributed simulation to the centralised simulation. A distributed solution can produce a significantly different final state compared to a centralised solution, despite handling all collisions correctly.

\subsection{Object state conflict}
The challenges involved with two servers reaching a consensus on the state of an object if the two objects have conflicting states will now be discussed. If a solution is used in which two servers simulate the same object (such as in an overlapping region between servers), diverging results from collisions can result in the same object can end up in different states on different servers. These conflicts in state may be small, in which case they can be resolved by either choosing the state from one server or taking the average of the two objects, however, this may appear as `jitter' if the differences between states is large enough. Furthermore, cases may exist when the two states have diverged significantly, such as the stack of cubes example in \ref{divergingResults}, in which case averaging of the two states cannot be used. One server's state would have to be chosen, which would appear on the other server as though the objects have `teleported' (instantaneously changed position). In addition to this, prior to the server having its object's states overwritten, those objects may have affected other objects in only that server. For example, if we continue with the stack of cubes example in \ref{divergingResults}, suppose the stack of cubes fell over on one server and one of the cubes collided with a sphere, that is only being simulated by that sever (for example, if it is outside of the overlapping region). The sphere is given a velocity as a result of the collision. If the cubes in the stack are then updated to be the same as on a remote server, essentially rewriting the history of the cubes, the end result would be a stack of cubes that is upright and sphere with a velocity from a collision that has been effectively erased from the simulation history.

\subsection{Late collisions}\label{lateCollisions}
Due to latency between servers, object positions on different servers will never be agreed upon exactly by both servers. The delay between servers means positions of objects on another server is always out of date. Even in solution where a single object is simulated on both servers, due to the non-determinism of real-time physics engines, the states will diverge. 

As a result of the out of date positions of objects, two objects hosted on different servers that would collide with each other, are unaware of the latest position of the other object. This means collisions between objects that are hosted on different servers will always occur late and at different positions on different servers, leading to differing results on different servers.

If objects that are going to collide are migrated between servers, they need to be migrated and arrive before a collision occurs. If the arrival takes place late, this may result in the two objects overlapping significantly (occupying the same space at the same time). The consequence of these late collisions are unstable collision response. To understand how late collisions lead to unstable collision response, we must first understand how in real-time collision detection works. In real-time physics engines, objects move in discrete steps, as a result of this, when two objects collide, they overlap each other i.e. penetrate. The collision is resolved by calculating the forces resulting from the collision and moving the objects so they no longer overlap (in PhysX if penetration is high, the latter may be done over several physics time-steps). If the penetration is high, such as in the case of a late collision response or if two objects are moving at high speed towards one another, the collision response can no longer guarantee stable results.

%In real-time physics, speed of calculation is favoured over accuracy of collision response, as a result of this 

\subsection{Missed collisions}
In cases of higher latency, or small or fast moving objects, collisions may be missed entirely. Similar to the bullet-through paper problem (where the discrete updating of objects means the two objects that should collide, never overlap and trigger a collision response), high latency can mean the state of a remote object can mean the state of the object is received so late that the local object that would collide, misses the object entirely. The larger or slower moving the objects involved are, the less likely this situation is to occur. Alternatively, if objects are migrated between servers, the object must arrive before the collisions happens, as discussed in \ref{lateCollisions}, if a delay in migration is severe enough, the collisions will be missed entirely.

\subsection{Missed updates}
Objects not being simulated (missed updates), can occur when objects are migrating between servers. When objects are being migrated between servers, they don't exist on any server, as a result the position and rotation of the object will not be updated. This would appear as though the object has frozen, how noticeable this would be to an observer would depend on the situation and the number of updates missed. For example, if the object is stationary or moving slowly, several time steps could be missed without be noticeable. Alternatively, in a scenario where a large object was in free-fall, very few update steps could be missed before an observer would notice that the object is not being simulated correctly.

\subsection{Object duplication or loss}
Any solution in which objects are migrated between servers is vulnerable to object duplication or loss. Networks do not provide absolute guarantee of message deliveries or order of message arrival. As a consequence of this, network messages for the migration of objects may be lost, in which case the object could be lost, if one server send a migrate message for an object and deletes the object from its simulation. Alternatively, a server may not delete the migrating object until an acknowledgement is received from the server it is migrating the object to, in which case, the acknowledgement may be lost and the object kept on the sending server, despite a duplicate of the object now existing on the receiving server. %Messages may arrive out of order

%Maximum penetration for a given speed occurs when two objects travelling directly towards each other have, in the final time-step before colliding the two objects have no distance between them, but are not overlapping, so not colliding. The next physics time-step, the collision will be detected and will have the maximum penetration possible for that relative speed. Re-arranging $s=d/t$, to $d=s.t$ and substituting the physics time-step for $t$, gives us the maximum distance (penetration) for a relative speed, and we are able to plot the maximum expected penetration line.

%, this is known as the collision response (in real-time physics, speed of calculation is favoured over accuracy of collision response). High collision penetration e.g. if two objects are moving quickly towards one another, can lead to unstable collision response, which is undesirable.

\definecolor{wrwrwr}{rgb}{0.3803921568627451,0.3803921568627451,0.3803921568627451}
\definecolor{dtsfsf}{rgb}{0.8274509803921568,0.1843137254901961,0.1843137254901961}
\definecolor{sexdts}{rgb}{0.1803921568627451,0.49019607843137253,0.19607843137254902}
\definecolor{rvwvcq}{rgb}{0.08235294117647059,0.396078431372549,0.7529411764705882}
\begin{figure}
	\centering
	\begin{tikzpicture}[line cap=round,line join=round,>=triangle 45,x=1cm,y=1cm, scale=2.4]
	%	\clip(-3.8316065591348587,-1.5915496679322039) rectangle (4.0193993686026035,3.119053888710274);
	\draw [line width=2pt,color=sexdts,fill=sexdts,fill opacity=0.6] (0.11306222137602934,0.5749956216017253) circle (0.2191629059122392cm);
	\fill[line width=2pt,color=dtsfsf,fill=dtsfsf,fill opacity=0.6] (-0.19130495931412153,2.497904290969918) -- (0.09727089010996666,2.49668178609732) -- (0.09849339498256493,2.785257635521408) -- (-0.19008245444152327,2.7864801403940063) -- cycle;
	\draw [line width=2pt,color=sexdts,fill=sexdts,fill opacity=0.6] (0.2690682445566622,2.8641372386638633) circle (0.18190589696740675cm);
	\fill[line width=2pt,color=dtsfsf,fill=dtsfsf,fill opacity=0.6] (-0.18001743348774918,1.688638597893834) -- (0.12779907363732385,1.6894047421549492) -- (0.12703292937620878,1.997221249280022) -- (-0.1807835777488642,1.996455105018907) -- cycle;
	\draw [line width=2pt,color=sexdts,fill=sexdts,fill opacity=0.6] (0.3506985149669561,2.1504628052606924) circle (0.20094089799255885cm);
	\fill[line width=2pt,color=rvwvcq,fill=rvwvcq,fill opacity=0.65] (-0.336221735744061,-0.21399083968233576) -- (-0.08697438199196733,-0.21915612145269675) -- (-0.08966346659279195,0.37774900628071006) -- (-0.336221735744061,0.37774900628071006) -- cycle;
	\draw [line width=2pt,color=dtsfsf,fill=dtsfsf,fill opacity=0.6] (-2.7581115521258512,-0.5210040634712074) circle (0.18765298545020567cm);
	\draw [line width=2pt,color=dtsfsf,fill=dtsfsf,fill opacity=0.6] (-2.4332540697970817,-0.7399608687412594) circle (0.1836317971285386cm);
	\draw [line width=2pt,color=dtsfsf,fill=dtsfsf,fill opacity=0.6] (-2.7438006498206193,-0.983246207930206) circle (0.20084779094096714cm);
	\fill[line width=2pt,color=dtsfsf,fill=dtsfsf,fill opacity=0.6] (-1.8,-0.3) -- (-1.5507526462479064,-0.30516528177036095) -- (-1.553441730848731,0.29173984596304586) -- (-1.8,0.29173984596304586) -- cycle;
	\draw [line width=2pt,color=dtsfsf,fill=dtsfsf,fill opacity=0.6] (-1.519067473250276,-0.8075899850392314) circle (0.2110032461078136cm);
	\draw [line width=2pt,color=dtsfsf,fill=dtsfsf,fill opacity=0.6] (-1.3242760580159627,-0.35254200026346094) circle (0.220457331048815cm);
	\fill[line width=2pt,color=rvwvcq,fill=rvwvcq,fill opacity=0.6] (1.6172164649279959,-0.10848935040022167) -- (1.9441305251416572,-0.10779885294426608) -- (1.9434400276857018,0.21911520726939515) -- (1.6165259674720405,0.21842470981343975) -- cycle;
	\draw [line width=2pt,color=sexdts,fill=sexdts,fill opacity=0.6] (2.1773098805776194,-0.021729373301184463) circle (0.19660778801195347cm);
	\fill[line width=2pt,color=sexdts,fill=sexdts,fill opacity=0.6] (2.3730665457464766,-0.11747262438789227) -- (2.7159883030391594,-0.11287162500071371) -- (2.7113873036519816,0.23005013229196902) -- (2.3684655463592983,0.2254491329047909) -- cycle;
	
	% Boundary Lines
	\draw [line width=1pt, dash pattern=on 5pt off 5pt] (-3,0)-- (0,0);
	\draw [line width=1pt, dash pattern={on 4pt off 3pt on 1pt off 3pt}] (0,3)-- (0,0);
	\draw [line width=1pt, dash pattern={on 2pt off 2pt}] (0,0)-- (3,0);
	
	% Server Labels
	\draw (-1.5,3.5) node[anchor=center] {Server 0};
	\draw (1.5,3.5) node[anchor=center] {Server 1};
	\draw (0,-1.0) node[anchor=center] {Server 2};
	
	\draw [line width=2pt,dash pattern=on 1pt off 1pt] (0.11306222137602934,0.5749956216017253) circle (0.33970250416397074cm);
	\draw [line width=2pt,dash pattern=on 1pt off 1pt] (0,1.824163562143371) circle (0.3724807823065095cm);
	\draw [line width=2pt,color=dtsfsf] (-0.19130495931412153,2.497904290969918)-- (0.09727089010996666,2.49668178609732);
	\draw [line width=2pt,color=dtsfsf] (0.09727089010996666,2.49668178609732)-- (0.09849339498256493,2.785257635521408);
	\draw [line width=2pt,color=dtsfsf] (0.09849339498256493,2.785257635521408)-- (-0.19008245444152327,2.7864801403940063);
	\draw [line width=2pt,color=dtsfsf] (-0.19008245444152327,2.7864801403940063)-- (-0.19130495931412153,2.497904290969918);
	\draw [line width=2pt,color=dtsfsf] (-0.18001743348774918,1.688638597893834)-- (0.12779907363732385,1.6894047421549492);
	\draw [line width=2pt,color=dtsfsf] (0.12779907363732385,1.6894047421549492)-- (0.12703292937620878,1.997221249280022);
	\draw [line width=2pt,color=dtsfsf] (0.12703292937620878,1.997221249280022)-- (-0.1807835777488642,1.996455105018907);
	\draw [line width=2pt,color=dtsfsf] (-0.1807835777488642,1.996455105018907)-- (-0.18001743348774918,1.688638597893834);
	\draw [line width=2pt,color=rvwvcq] (-0.336221735744061,-0.21399083968233576)-- (-0.08697438199196733,-0.21915612145269675);
	\draw [line width=2pt,color=rvwvcq] (-0.08697438199196733,-0.21915612145269675)-- (-0.08966346659279195,0.37774900628071006);
	\draw [line width=2pt,color=rvwvcq] (-0.08966346659279195,0.37774900628071006)-- (-0.336221735744061,0.37774900628071006);
	\draw [line width=2pt,color=rvwvcq] (-0.336221735744061,0.37774900628071006)-- (-0.336221735744061,-0.21399083968233576);
	\draw [line width=2pt,dash pattern=on 1pt off 1pt] (-0.20008010870040147,0.05759193325942933) circle (0.4395227390987423cm);
	\draw [line width=2pt,dash pattern=on 1pt off 1pt] (-2.7581115521258512,-0.5210040634712074) circle (0.3403858079791503cm);
	\draw [line width=2pt,dash pattern=on 1pt off 1pt] (-2.4332540697970817,-0.7399608687412594) circle (0.3064438391841357cm);
	\draw [line width=2pt,dash pattern=on 1pt off 1pt] (-2.7438006498206193,-0.983246207930206) circle (0.32740392092189946cm);
	\draw [line width=2pt,color=dtsfsf] (-1.8,-0.3)-- (-1.5507526462479064,-0.30516528177036095);
	\draw [line width=2pt,color=dtsfsf] (-1.5507526462479064,-0.30516528177036095)-- (-1.553441730848731,0.29173984596304586);
	\draw [line width=2pt,color=dtsfsf] (-1.553441730848731,0.29173984596304586)-- (-1.8,0.29173984596304586);
	\draw [line width=2pt,color=dtsfsf] (-1.8,0.29173984596304586)-- (-1.8,-0.3);
	\draw [line width=2pt,dash pattern=on 1pt off 1pt] (-1.6811522710982656,0.015633897433233776) circle (0.4555571106906273cm);
	\draw [line width=2pt,dash pattern=on 1pt off 1pt] (-1.519067473250276,-0.8075899850392314) circle (0.40351234312900314cm);
	\draw [line width=2pt,dash pattern=on 1pt off 1pt] (-1.3242760580159627,-0.35254200026346094) circle (0.3214616358035378cm);
	\draw [line width=2pt,color=rvwvcq] (1.6172164649279959,-0.10848935040022167)-- (1.9441305251416572,-0.10779885294426608);
	\draw [line width=2pt,color=rvwvcq] (1.9441305251416572,-0.10779885294426608)-- (1.9434400276857018,0.21911520726939515);
	\draw [line width=2pt,color=rvwvcq] (1.9434400276857018,0.21911520726939515)-- (1.6165259674720405,0.21842470981343975);
	\draw [line width=2pt,color=rvwvcq] (1.6165259674720405,0.21842470981343975)-- (1.6172164649279959,-0.10848935040022167);
	\draw [line width=2pt,color=sexdts] (2.3730665457464766,-0.11747262438789227)-- (2.7159883030391594,-0.11287162500071371);
	\draw [line width=2pt,color=sexdts] (2.7159883030391594,-0.11287162500071371)-- (2.7113873036519816,0.23005013229196902);
	\draw [line width=2pt,color=sexdts] (2.7113873036519816,0.23005013229196902)-- (2.3684655463592983,0.2254491329047909);
	\draw [line width=2pt,color=sexdts] (2.3684655463592983,0.2254491329047909)-- (2.3730665457464766,-0.11747262438789227);
	\draw [line width=2pt,dash pattern=on 1pt off 1pt] (1.7707737818301332,0.042356814063601776) circle (0.3324655775151698cm);
	\draw [line width=2pt,dash pattern=on 1pt off 1pt] (2.5485828739333973,0.0529545076040507) circle (0.31587509276975684cm);
	
	% Object labels
	\draw[color=dtsfsf] (-0.5,3) node {A};
	\draw[color=sexdts] (0.75,3) node {B};
	\draw[color=dtsfsf] (-0.6,1.9) node {C};
	\draw[color=sexdts] (.75,1.9) node {D};
	\draw[color=dtsfsf] (-1.66,0.75) node {E};
	\draw[color=dtsfsf] (-0.85,-0.35) node {F};
	\draw[color=dtsfsf] (-1.5,-1.45) node {G};
	\draw[color=dtsfsf] (-3.3,-0.5) node {H};
	\draw[color=dtsfsf] (-2.2,-1.2) node {I};
	\draw[color=dtsfsf] (-3.3,-1) node {J};
	\draw[color=sexdts] (2.2,-0.5) node {K};
	\draw[color=sexdts] (2.5,0.5) node {L};
	\draw[color=rvwvcq] (1.8,-0.5) node {M};
	\draw[color=rvwvcq] (-0.6,0.6) node {N};
	\draw[color=sexdts] (0.5,1) node {O};
	
	\end{tikzpicture}
	
	\begin{tikzpicture}[line cap=round,line join=round,>=triangle 45,x=1cm,y=1cm, scale=0.75, every node/.style={font=\scriptsize,
		minimum height=0.25cm,minimum width=0.5cm},scale=0.75]
	\clip(-8,-0.05) rectangle (4,2.15);
	
	% Boundary Lines
	\draw [line width=1pt, dash pattern={on 4pt off 3pt on 1pt off 3pt}] (-7.5,1.8)-- (-6.5,1.8);
	\draw (-6,1.8) node[anchor=west] {0-1 Boundary};
	\draw [line width=1pt, dash pattern=on 5pt off 5pt] (-7.5,1.3)-- (-6.5,1.3);
	\draw (-6,1.3) node[anchor=west] {0-2 Boundary};
	\draw [line width=1pt, dash pattern={on 2pt off 2pt}] (-7.5,0.8)-- (-6.5,0.8);
	\draw (-6,0.8) node[anchor=west] {1-2 Boundary};
	
	\draw [line width=2pt] (-7.5,0.3)-- (-6.5,0.3);
	\draw (-6,0.3) node[anchor=west] {Aura};
	
	% Object boxes
	\fill[line width=2pt,color=sexdts,fill=sexdts,fill opacity=0.6] (0.2,0.8) -- (0.7,0.8) -- (0.7,1.3) -- (0.2,1.3) -- cycle;
	\fill[line width=2pt,color=rvwvcq,fill=rvwvcq,fill opacity=0.6] (0.2,-0.0) -- (0.7,-0.0) -- (0.7,0.5) -- (0.2,0.5) -- cycle;
	\fill[line width=2pt,color=dtsfsf,fill=dtsfsf,fill opacity=0.6] (0.2,1.6) -- (0.7,1.6) -- (0.7,2.1) -- (0.2,2.1) -- cycle;
	
	\draw [line width=2pt,color=sexdts] (0.2,0.8)-- (0.7,0.8);
	\draw [line width=2pt,color=sexdts] (0.7,0.8)-- (0.7,1.3);
	\draw [line width=2pt,color=sexdts] (0.7,1.3)-- (0.2,1.3);
	\draw [line width=2pt,color=sexdts] (0.2,1.3)-- (0.2,0.8);
	
	\draw [line width=2pt,color=rvwvcq] (0.2,-0.0)-- (0.7,-0.0);
	\draw [line width=2pt,color=rvwvcq] (0.7,-0.0)-- (0.7,0.5);
	\draw [line width=2pt,color=rvwvcq] (0.7,0.5)-- (0.2,0.5);
	\draw [line width=2pt,color=rvwvcq] (0.2,0.5)-- (0.2,-0.0);
	
	\draw [line width=2pt,color=dtsfsf] (0.2,1.6)-- (0.7,1.6);
	\draw [line width=2pt,color=dtsfsf] (0.7,1.6)-- (0.7,2.1);
	\draw [line width=2pt,color=dtsfsf] (0.7,2.1)-- (0.2,2.1);
	\draw [line width=2pt,color=dtsfsf] (0.2,2.1)-- (0.2,1.6);
	
	\draw (1.0,2.2) node[anchor=north west] {Server 0 Object};
	\draw (1.0,1.4) node[anchor=north west] {Server 1 Object};
	\draw (1.0,0.6) node[anchor=north west] {Server 2 Object};
	
	\end{tikzpicture}
	\caption{Aura Projection Scenarios}
	\label{AuraProj}
\end{figure}

\section{Naive Approach}\label{CrossBoundaryInteraction}
To provide a comparison and to demonstrate our contribution to the field, a naive approach is described first. The naive approach is used to highlight the challenges faced when solving this problem.

Assuming a migratory approach in which servers are responsible for a geographic area, when one object crosses a boundary between geographic areas, such an object's hosting must be transferred to the appropriate server. It is handling this transfer that is the underlying problem.

Accomplishing a transfer with a simple message from one server to another is charged with many problems: (1) the message may be lost and the object disappears; (2) the message may be delayed and the object disappears and reappears; (3) there may be more than one candidate server for hosting (if object travelling quickly) resulting in duplication of an object; (4) the server an object leaves may be unaware of when it should stop simulating the object (resulting in duplicated objects)\cite{High-performanceServerSystems}; (5) an object may be colliding with another object across the boundary which would be lost if both objects are hosted on different servers (objects may pass through each other).

For additional clarity consider objects A and B in Fig. \ref{AuraProj}. Each server is aware of the objects they host but not other objects. As such, objects A and B would not interact and not collide with each other, creating a time-space inconsistency. A time-space inconsistency is defined as two objects occupying the same space at the same time. If Object B has a velocity directed towards the boundary, it would continue at its initial velocity until it traversed the boundary, at which point it would migrate to server 0. Upon completion of the migration, objects A and B would overlap far more than would be expected in a typical simulation, resulting in inappropriate collision response.

\section{Considered Solutions}\label{ConsideredSolutions}
There are several solutions that should be considered to distribute a real-time physics simulation across multiple machines over a network:

\subsection{Both servers simulate objects on boundary}
Simulate objects overlapping the boundary on both servers, however, results would diverge significantly as objects interact with objects not overlapping the boundary and there is the extra computational overhead of needing to be simulate objects twice.

\subsection{Both servers simulate objects in an overlapping region}
Simulate objects overlapping the boundary and objects interacting with those on both servers, e.g. by using an overlapping region. Object may interact with objects not in the overlapping region, so the results would diverge or the overlapping region would have to dynamically change in order to keep all interacting objects within the overlapping region. Even if all objects interacting were in the overlapping region, results would still diverge due to non-determinism of physics engines. The divergence of results due to non-determinism could be corrected by periodically syncing the objects (ensuring the objects have the same state on both servers), however, this would appear as jitter as objects would be moved from one position to their corrected position instantaneously. In addition, this would require additional network overhead for the syncing of objects between servers and extra computational overhead of needing to simulate objects twice.

There would be a trade-off with this approach, either:
1) Frequent messages are sent with small corrections for objects, appearing as jitter. This would require lots of computational and network overhead. 
2) Infrequent messages with large corrections for objects, appearing as objects moving large distances instantaneously but would require less computational and network overhead.

For new objects entering the overlapping region, there will be a delay before the new object appears on both servers, network delays would affect this and could lead to the two simulations diverging significantly. For example, if there are two fast moving objects on a collision course with each other that are expected to collide in the overlapping region. Due to network delays, each server only becomes aware of the other object once the objects have already passed each other, as a result the collision is missed entirely and the result is incorrect compared to a single simulation.
%For example, a newly entered object from one server knocked over a stack of cubes in the overlapping region, if there was a long delay in communication, the receiving server might one physics timestep be unaware of any collision, the next timestep it would update to many timesteps after the collision

\subsection{Over network object collisions}
Objects in the overlapping region interact over the network. This would be done with the two physics in lock-step, i.e. neither progresses until all computation divided between the servers has been completed for that physics step. Servers could divide the computation of collision detection and/or response between servers and then exchange results. This solution would have to simulate objects twice and would require multiple round-trips of messages over the network per physics time-step, but the collision response for objects would be identical on both servers. This approach would be affected by network delays significantly, as not only are multiple round-trips of messages required between server each physics time-step, the simulation won't advance until all messages for that physics time-step have been completed, meaning the frequency of the simulation would be negatively affected.

% All these solutions require fast and reliable network connections, except AP

\subsection{Remote objects as static}
Treat objects being simulated on another server as static. However, this would not lead to correct collision response, if compared to a single simulation. Momentum is conserved in collisions and no momentum is transferred to static objects, therefore the resulting momentum on the colliding object would not be the same as if the two objects were dynamic (not static) on a single simulation.
Half of all collisions would be missed, as collisions would be resolved for the object not being treated as static, but on the remote server, the collision never appears to happen, as the collision has already been resolved when receiving the updated state of the interacting object.
This could also lead to jitter, if two objects are in contact (i.e. colliding over multiple frames, such as a box stacked on top of another box), the server in which the collision is resolved on could be different every physics step. An cube stacked on top of a static cube may have a different outcome to a cube with a static cube stacked on top of it.
% It is probably more complicated than this, as physics engines do a lot to deal with objects in contact, constantly moving a static object is probably going to interfere with this.

% Collisions may not be missed if there is deep penetration as PhysX won't move objects all the way out of a collision if the collision is deep enough.

This approach would be affected by network delays. For example, using the stacked cubes example. Say the two cubes are free-falling. If the bottom cube was being simulated on a different server, the position of the bottom cube would only be updated whenever a message was received from the remote server to do so. If there was a long delay between messages from the remote server, for example, from a spike in latency or a sudden big loss in packets, the cube would not be updated and appear to have stopped falling. As the bottom cube has stopped on the receiving server, the top cube remains stationary, as it is in contact with the bottom cube. On the remote server, if messages are being received, the cube would be stationary and appear to be floating, as from its server's perspective, it is in contact with the bottom cube. As a result, both servers have incorrect states compared to a single simulation.


\subsection{All object interactions on same server}
Simulate all objects that are interacting with each other on the same server. Auras could be used to determine which objects are potentially interacting. Auras of objects would need to be exchanged. Solution would need to manage number of auras exchanged to reduce this overhead.

\section{Proposed Solution}
% Introduce this in context of above section
We term our approach Aura Projection (AP). AP tackles the problem of maintaining consistency across regions while ensuring timeliness of a simulation. In particular, AP provides the building blocks of a scalable solution to server-side physics simulations by handling all configurations of boundary cases in migratory approaches. 

In AP, an object maintains an aura, an area of interest around the object, that indicates its possible future location and, therefore, aids in identifying near future interactions. In other words, auras act as a request for objects to be migrated. Any object colliding with the aura is migrated to the server the aura came from. The presence of an aura allows: (1) the prediction of future hosting requirements allowing time to transfer objects; (2) a narrowing of interest in only considering a subset of objects promoting a scalable solution; (3) an unhindered physics simulation for existing objects; (4) limiting communications requirements between servers based on focusing message passing overhead on interacting objects (promoting scalability). In essence, AP ensures that any two objects that may be interacting are always being simulated on the same server. For example, objects C and D illustrated in Fig. \ref{AuraProj}. Object C projects its aura into Server 1, Object D collides with the aura and is migrated to Server 0, allowing continued interaction to occur.

Messages are sent between servers where aura overlap occurs across a geographic boundary. We term this phase of the algorithm as an object projecting its aura onto another server. %The receiving server can then predict and prepare for the possibility of interaction if an object was migrated to it. 

In the context of real-time physics simulations, auras appear on the receiving server as a trigger volume (non-physically interacting volume). For each simulation step that an object is projecting an aura, the object's position and calculated aura are sent. As a network optimisation, if the boundary object is not moving, no details are sent and the aura on at a receiving server is unaltered.

When an aura is no longer being projected for an object (boundary not overlapped), the server simulating that object sends a message to any servers receiving the aura to remove the aura of that object.

In summary, objects that are being hosted by a remote server are consistency-aware, through the use of auras. The system doesn't know exactly where the object will be, but can guarantee it will be within the aura by the time an interaction takes place.
A small amount of availability is sacrificed (in the region of 10ms) as objects migrate between servers, in order to achieve eventual consistency while maintaining network partitioning.
In terms of collisions between objects, AP provides strong consistency. All collisions between objects only take place on a single server, removing the possibility of inconsistencies between servers. Network partitioning is sacrificed in order to achieve strong consistency.% and no sacrifice to availability is made.

\section{Aura Calculation}
Calculating an aura size is key to creating a scalable solution balanced against consistency: (1) Auras that are too small will result in a faster simulation but with more missed interactions; (2) Auras that are too large will include more calculations that are simply not required, increasing server load and decreasing scalability. Therefore, how auras are calculated will be carefully described and justified. 

Spheres are used to represent auras. This bounding volume is computationally efficient as rotational calculations are not required in determining its correct alignment. Auras accommodate the displacement of objects using an estimated network latency based on historic monitoring, in addition to the distance a remote object may be predicted to penetrate an aura (considering velocities).

Rather than every object maintaining an aura, which would add overhead to the physics simulation, when an aura is exchanged it accounts for a potential remote object's aura. When an object comes into the aura distance of the region boundary, an aura is created for the object and the aura starts being broadcast to neighbouring servers.

Fig. \ref{sequence_diagram} demonstrates the sequence of events involved when an aura is sent, received, and collided with, resulting in an object migration.

To lower inconsistencies during migration, the following user-defined tolerances are used for the aura calculations: 
\begin{itemize}
	\item $V_{t}$ - maximum speed of simulated objects.
	\item $T_{L}$ - maximum network latency time.
	\item $T_{F}$ - maximum frame-time (time allowed for combined frame update and network update).
\end{itemize}
If velocities, latencies, or frame-time are above these tolerances, then stability is no longer guaranteed. This enables AP to deterministically indicate to the overall simulation when latency may be influencing the mutually consistent views of the servers, which in turn manifests as errors in the physics.

To calculate the radius of an aura, the maximum distances travelled by objects within the delay time between simulations must first be calculated. To calculate the maximum distance an object may travel for a given time, the following formula is used: $\triangle s=v\cdot \triangle t$, where $ \triangle s$ is distance, $v$ is speed, and $ \triangle t$ is time. In this context, $ \triangle v$ is substituted with maximum speed tolerance, and $ \triangle t$ will be a multiple of the physics step time (to accommodate discrete time step calculations in the physics engine). $ \triangle t$ can be calculated using the maximum frame-time and maximum latency tolerance, discussed below.

There is a time delay between the aura being created on a host and being created on a receiving server, which is made up of: up to one frame before sending the aura creation message; inter-server latency; up to one frame from the message being received and acted upon; the time delay between the aura being created and the detection of the collision between a potential remote object and the aura by the physics update step. This last time delay is accounted for by rounding the previous delays up to the nearest physics time step.

Using the time delays mentioned above and substituting the relevant tolerances, the maximum displacement time ($Max_{DT}$) of a remote object is calculated using the following equation:
\DeclarePairedDelimiter\ceil{\lceil}{\rceil}
\begin{equation}
T_{R}=\ceil*{\frac{2\cdot T_{F} + T_{L}}{T_{P}}}T_{P}
\end{equation}

$T_{R}$ is the $Max_{DT}$ of a remote object, $T_{F}$ is the frame-time tolerance, $T_{L}$ is the latency tolerance and $T_{P}$ is the physics step time.

The aura has to account for the maximum displacement of both the object on the host server and a potential object on the remote server. The total $Max_{DT}$ will be the sum of: a) the $Max_{DT}$ of a remote object, subtracting one physics time step, as the aura only needs to account for the displacement of the host object after the creation of the aura; b) The following time delays: a time delay of up to one frame time between the physics step of the remote server and the remote server update loop sending the migration message from the migration buffer; a time delay of the latency between servers; a time delay of up to one frame time between the message being received by a host and a host acting on the migration message and creating the migrated object in the physics engine; a delay between the migrated object being created and the collision being detected by the physics update step. This last time delay is accounted for by rounding the previous delays up to the nearest physics time step.

The $Max_{DT}$ of the object on the host that is projecting the aura is therefore calculated using the following equation:
\begin{equation}
T_{H}=T_{R}-T_{P}+\ceil*{\frac{2\cdot T_{F} + T_{L}}{T_{P}}}T_{P}
\end{equation}
$T_{H}$ is the $Max_{DT}$ of the object on the host.

The total $Max_{DT}$ is therefore $T_{R}+T_{H}$, which can be simplified to the following equation:
\begin{equation}
T_{T}=(3\ceil*{\frac{2\cdot T_{F} + T_{L}}{T_{P}}}-1)T_{P}
\end{equation}
$T_{T}$ is the total $Max_{DT}$.

The aura of an object can then be calculated using the following equation: 
\begin{equation}\label{auraEquation}
R_{a}=R_{o}+(V_{t}\cdot T_{T})
\end{equation}
$R_{a}$ is the radius of the aura in $m$, $R_{o}$ is the bounding sphere of the object, $V_{t}$ is the speed tolerance in $m\mathord{\cdot}s^{-1}$ and $T_{T}$ is the total $Max_{DT}$ in $s$.

\begin{figure}[!t]	
	\begin{tikzpicture}[line cap=round,line join=round,>=triangle 45,x=1cm,y=1cm,every node/.style={font=\scriptsize,
		minimum height=0.25cm,minimum width=0.5cm},,scale=1.0]

\draw [line width=1pt,dash pattern=on 5pt off 5pt] (-2.0,6)-- (-2.0,-7.25);
\draw [line width=1pt,dash pattern=on 5pt off 5pt] (2.0,6)-- (2.0,-7.25);
\draw [line width=1pt,dash pattern=on 5pt off 5pt] (6.0,6.0)-- (6.0,-7.25);
\draw [line width=1pt,dash pattern=on 5pt off 5pt] (-6.0,6)-- (-6.0,-7.25);
\draw [->,line width=1pt] (-5.5,4) -- (-2.0,4);
\draw [->,line width=1pt] (-1.5,2.5) -- (2.0,1.5);
\draw [->,line width=1pt] (2.5,0) -- (6.0,0);
\draw [->,line width=1pt] (5.5,-1.5) -- (2.0,-1.5);
\draw [->,line width=1pt] (1.5,-3) -- (-2.0,-4);
\draw [->,line width=1pt] (-2.5,-5.5) -- (-6.0,-5.5);
\fill[line width=2pt,color=blue!20] (-6.5,5) -- (-6.5,4) -- (-5.5,4) -- (-5.5,5) -- cycle;
\fill[line width=2pt,color=blue!20] (-2.5,3.5) -- (-2.5,2.5) -- (-1.5,2.5) -- (-1.5,3.5) -- cycle;
\fill[line width=2pt,color=blue!20] (1.5,1) -- (1.5,0) -- (2.5,0) -- (2.5,1) -- cycle;
\fill[line width=2pt,color=blue!20] (5.5,-0.5) -- (5.5,-1.5) -- (6.5,-1.5) -- (6.5,-0.5) -- cycle;
\fill[line width=2pt,color=blue!20] (1.5,-2) -- (1.5,-3) -- (2.5,-3) -- (2.5,-2) -- cycle;
\fill[line width=2pt,color=blue!20] (-2.5,-4.5) -- (-2.5,-5.5) -- (-1.5,-5.5) -- (-1.5,-4.5) -- cycle;
\fill[line width=2pt,color=blue!20] (-6.5,-6) -- (-6.5,-7) -- (-5.5,-7) -- (-5.5,-6) -- cycle;
\draw [line width=1pt] (-6.5,5)-- (-6.5,4);
\draw [line width=1pt] (-6.5,4)-- (-5.5,4);
\draw [line width=1pt] (-5.5,4)-- (-5.5,5);
\draw [line width=1pt] (-5.5,5)-- (-6.5,5);
\draw [line width=1pt] (-2.5,3.5)-- (-2.5,2.5);
\draw [line width=1pt] (-2.5,2.5)-- (-1.5,2.5);
\draw [line width=1pt] (-1.5,2.5)-- (-1.5,3.5);
\draw [line width=1pt] (-1.5,3.5)-- (-2.5,3.5);
\draw [line width=1pt] (1.5,1)-- (1.5,0);
\draw [line width=1pt] (1.5,0)-- (2.5,0);
\draw [line width=1pt] (2.5,0)-- (2.5,1);
\draw [line width=1pt] (2.5,1)-- (1.5,1);
\draw [line width=1pt] (5.5,-0.5)-- (5.5,-1.5);
\draw [line width=1pt] (5.5,-1.5)-- (6.5,-1.5);
\draw [line width=1pt] (6.5,-1.5)-- (6.5,-0.5);
\draw [line width=1pt] (6.5,-0.5)-- (5.5,-0.5);
\draw [line width=1pt] (1.5,-2)-- (1.5,-3);
\draw [line width=1pt] (1.5,-3)-- (2.5,-3);
\draw [line width=1pt] (2.5,-3)-- (2.5,-2);
\draw [line width=1pt] (2.5,-2)-- (1.5,-2);
\draw [line width=1pt] (-2.5,-4.5)-- (-2.5,-5.5);
\draw [line width=1pt] (-2.5,-5.5)-- (-1.5,-5.5);
\draw [line width=1pt] (-1.5,-5.5)-- (-1.5,-4.5);
\draw [line width=1pt] (-1.5,-4.5)-- (-2.5,-4.5);
\draw [line width=1pt] (-6.5,-6)-- (-6.5,-7);
\draw [line width=1pt] (-6.5,-7)-- (-5.5,-7);
\draw [line width=1pt] (-5.5,-7)-- (-5.5,-6);
\draw [line width=1pt] (-5.5,-6)-- (-6.5,-6);
\draw[color=black,align=center] (-4,3.0) node {Boundary\\Collision};
\draw[color=black,align=center] (4.2,0.9) node {Create\\Aura};
\draw[color=black,align=center] (0.0,2.9) node {Send\\Aura};
\draw[color=black,align=center] (4.0,-2.5) node {Aura\\Collision};
\draw[color=black,align=center] (-0.0,-2.5) node {Migrate\\Object};
\draw[color=black,align=center] (-4.0,-4.5) node {Create\\Migrated\\Object};

\draw (-6,6) node[draw,anchor=center,align=center,fill=white] {Server 0\\Physics Update};
\draw (-2,6) node[draw,anchor=center,align=center,fill=white] {Server 0\\Update Loop};
\draw (6,6) node[draw,anchor=center,align=center,fill=white] {Server 1\\Physics Update};
\draw (2,6) node[draw,anchor=center,align=center,fill=white] {Server 1\\Update Loop};
\end{tikzpicture}
	\caption{Aura and migration sequence diagram}
	\label{sequence_diagram}
\end{figure}

\section{Sub-optimal object hosting}
As objects within each others auras need to be simulated on the same server, this can lead to objects being deep within another server's region (objects that are far from the region boundary).

Clusters of objects can therefore end up being simulated with all but one of their objects in their host server's region. This is sub-optimal, as the more auras that are being maintained/exchanged, the higher the computational cost.

\section{Thrashing}\label{Thrashing}
The common term `thrashing' will be used to describe the following two scenarios that must be considered.

1. An object may be overlapping two auras from different servers. Given appropriate velocity, this could result in object migration, followed by aura collision, followed by migration and then repeating the process again. For example, Objects K, L and M in Fig. \ref{AuraProj}. Object K lies inside the auras of both Object L and M. 

In order to prevent this thrashing, when an object is migrated, it is also migrated with any objects found that lie within the aura of the object or have an existing aura which overlaps the migrating object's aura. This is carried out recursively to ensure all objects that are overlapping are migrated at the same time.

2. Two objects on different servers may both trigger the migration of the other. For example, Objects C and D in Fig. \ref{AuraProj}. Suppose the two objects are moving towards each other. Object C collides with the aura of object D and begins migrating to Server 1. Before Object C is received (and the message processed), object D collides with the aura of Object C and begins migrating to Server 0. The two objects remain on different servers and the process can repeat again, indefinitely.

When Object C is received, the object is created in Server 0's physics simulation and the aura for that object is removed from Server 0's physics simulation. 
The next physics time-step on Server 0, the object is within the aura distance of the boundary with Server 1, a message for the aura of object C will added to the send buffer for Server 1. The next network update, the message will be sent to Server 1.
If Object C intersects an aura from Server 1, in this case Object D's aura, it will be added to the object send buffer for Server 1. The next network update, the message will be sent to Server 1. %And the aura no longer sent?
The message for Object C's aura is received by Server 1 and processed on the next network update, when processed, a trigger volume for Object C's aura will be created in the physics engine of Server 1. The next physics time-step on Server 1, any objects within that aura will call a trigger callback, in this case Object D, indicating that object D has collided with an aura and so should be migrated to Server 0. Object D is added to the object send buffer for Server 0 and the next network update, Object D is migrated.

%TODO: Check if aura sent in addition to the object migration? Yes, but it is also sent with an aura remove message
%TODO: Check if aura removed immediately when object is received? Will be received at the same time as the aura remove message, which will remove the aura immediately
%TODO: WIll the immediately remove aura still trigger a callback? It shouldn't do, as it checks for removed shapes


In order for this process to repeat, both objects must have their migrations triggered (during a physics update) before the message of the other object migration has been received and processed and a physics time step has occurred to remove the aura trigger volume from the physics simulation.

In order for this to occur, each object must have its aura created (first physics time step after arrival) before the aura of the other object or the other object itself has been received (if the aura is received, it will trigger the migration immediately, so an aura will not be sent, if the other object is received, the aura will be removed). Each object has its aura received by the other server, triggering the migration of the other object. The cycle is broken once an object is received and its aura removed before the other object has its migration triggered when a physics time step occurs.

%Proposed solution 1: If an object collides with an aura that has just been created (was created in the most recent physics time step), only migrate the object if the colliding object id is greater (or lower than) the object projecting the aura's id. This will allow a physics time step to occur and the aura removed, if the object projecting the aura is being migrated. Should work so long as the migration message is received and processed on the waiting server before its next physics time step, i.e. round-trip time is less than two physics time steps. Aura is sent on step 0, remote aura is received, it gets created in step 1 and ignored for the first step of existense, the migrate message is received and the aura is removed in step 2

%Proposed solution 2: If an object collides with another object's aura and is also projecting an aura they will both be colliding with each other's aura (if the two objects have identical rotations or are spheres). If an object knows the other object is colliding/will collide with its aura, only migrate if object id is higher or lower or rank of server is higher or lower. Would rely on knowing the exact shape of the collider of the other object

\section{Islands}
When an object traverses a region boundary, the object is migrated only if it does not overlap an aura projected by an object from the same server. In the context of migrations, an island is defined as two or more objects located outside of their host server's region (i.e. have traversed a region boundary) but are each within the aura projected from objects owned by the same host. For example, Objects H, I, and J illustrated in Fig. \ref{AuraProj}. No objects in the island are migrated as each object is within the aura of another object in the island. This should be prevented as it causes processing and networking overhead and is unnecessary as all objects lie within the region of Server 2 and are not interacting with any other objects from Server 0. 

To prevent islands, a search is performed at each time step to determine if an object is part of an island. A search is performed to determine if a potential migratory object is within a group of objects with overlapping auras, of which none are positioned within the hosting server region. If the group of objects has no members within the hosting server region then the object is part of an island and the entire island of objects is migrated. Otherwise, no action is taken. For example, in Fig. \ref{AuraProj} Object I has overlapping auras with Objects H and J. H and J are checked for overlap with the Server 0 - Server 1 region boundary and for overlap with auras from Server 0 that are intersecting the Server 0 - Server 1 boundary. In this scenario, H and J are found to be part of an island, so H, I, and J are all migrated to Server 2. If objects are found to not be part of an island, no action is taken. For example, Objects E, F, and G in Fig. \ref{AuraProj}. Object F is found to have an overlapping aura with Object E, but Object E intersects the Server 0 - Server 1 boundary, so E, F, and G are found to not be part of an island and therefore no action is taken. This solution is shown in Algorithm \ref{boundaryAlgorithm} and the problem of islands can therefore be considered solved.

\section{Corner Case}\label{CornerCase}
This section discusses how AP handles corner cases, i.e., where the boundary of more than two servers meet. 

An example of the corner case is illustrated in Fig. \ref{AuraProj}. Object N, hosted on Server 2, sends an aura to Server 0. Object O, on Server 1, sends an aura to Server 0. Object N's aura overlaps with Object O's on server 0. Server 2 is unaware of Object O and Server 1 is unaware of Object N. As the two auras overlap, there is a potential interaction between the objects, yet the two objects remain on separate servers. To solve this problem, auras from boundaries between two neighbouring servers are shared. In the region layout in Fig. \ref{AuraProj}, auras from a boundary are received by interested servers. In the Object N and O example, Server 2 would send Object N's aura to Server 1; when Object O collides with Object N's aura, migration to Server 2 occurs.

A server has to receive the auras of all objects from neighbouring boundaries, as objects being simulated by a server can exist anywhere in a neighbouring server's region. Our assumption is that a region a server simulates is sufficiently large enough to prevent the aura overlap of objects from non-neighbouring servers.
