% ************************** Thesis Abstract *****************************
% Use `abstract' as an option in the document class to print only the titlepage and the abstract.
\begin{abstract}
In this study, a solution to delivering scalable real-time physics simulations is proposed. Although high performance computing simulations of physics related problems do exist, these are not real-time and do not model the real-time intricate interactions of rigid bodies for visual effect common in video games (favouring accuracy over real-time). As such, this study presents the first approach to real-time delivery of scalable, commercial grade, video game quality physics and is termed Aura Projection (AP). This approach takes the physics engine out of the player's machine and deploys it across standard cloud based infrastructures. The simulation world is divided into regions that are then allocated to multiple servers. A server maintains the physics for all simulated objects in its region. The contribution of this study is the ability to maintain a scalable simulation by allowing object interaction across region boundaries using predictive migration techniques. AP allows each object to project an aura that is used to determine object migration across servers to ensure seamless physics interactions between objects. AP allows player interaction at any point in real-time (influencing the simulation) in the same manner as any video game.

This study measures and evaluates both the scalability of AP and correctness of collisions within AP through experimentation and benchmarking. The experiments show that AP is a solution to scalable real-time physics by measuring computation workload with increasing computation resources. AP also demonstrates that collisions between rigid-bodies can be simulated correctly within a scalable real-time physics simulation, even when rigid-bodies are intersecting server-region boundaries; demonstrated through comparison of a distributed AP simulation to a single, centralised simulation. 
We believe that AP is the first successful demonstration of scalable real-time physics in an academic setting.
\end{abstract}
