% ************************** Thesis Abstract *****************************
% Use `abstract' as an option in the document class to print only the titlepage and the abstract.
\begin{abstract}
In this thesis we propose a solution to delivering scalable real-time physics simulations. Although high performance computing simulations of physics related problems do exist, these are not real-time and do not model the real-time intricate interactions of rigid bodies for visual effect common in video games (favouring accuracy over real-time). As such, this paper presents the first approach to real-time delivery of scalable, commercial grade, video game quality physics. We have termed this approach Aura Projection (AP). This approach takes the physics engine out of the player's machine and deploying it across standard cloud based infrastructures. The simulation world is then divided into sections that are then allocated to servers. A server maintains the physics for all simulated objects in its section. Our contribution is the ability to maintain a scalable simulation by allowing object interaction across section boundaries using predictive migration techniques. We allow each object to project an aura that is used to determine object migration across servers to ensure seamless physics interactions between objects. Our approach allows player interaction at any point in real-time (influencing the simulation) in the same manner as any video game.
We measure and evaluate both the scalability of AP and correctness of collisions within AP through experimentation and benchmarking. Our experiments show that AP is a solution to scalable real-time physics by measuring computation workload with increasing computation resources. We also demonstrate that collisions between rigid-bodies can be simulated correctly within a scalable real-time physics simulation, even when rigid-bodies are intersecting server-region boundaries; demonstrated through comparison of a distributed AP simulation to a single, centralised simulation.
%The validity of our results is demonstrated through experimentation and benchmarking. 
 We believe that this is the first successful demonstration of scalable real-time physics.
\end{abstract}
